\documentclass[11pt]{article}
\usepackage[utf8]{inputenc}
\usepackage[T1]{fontenc}     % Fix weird character
\usepackage{geometry}
\usepackage{amssymb}
\usepackage{amsmath}
\usepackage{gensymb}
\usepackage{parskip}
\usepackage[style=ieee,backend=biber]{biblatex}
\usepackage[breaklinks=true,bookmarks=true,hidelinks]{hyperref}
\usepackage{tikz}

\geometry{
    a4paper,
    hmargin=2.54cm,
    tmargin=1.27cm,
    bmargin=1.27cm,
    includeheadfoot
}
\setcounter{secnumdepth}{0}  % Disable section numbering

\begin{document}

\section{Elementära funktioner}

\subsection{Logaritmfunktioner}

\textbf{Egenskaper hos logaritmer} (p.132)

\begin{align}
    {}^a\log a^x &= x, \qquad x \in \mathbb{R},\\
    a^{\log x} &= x, \qquad x > 0,\\
    {}^a\log x &= \frac{\log{x}}{\log{a}}.
\end{align}

\textbf{Sats 8.4} \textit{För $x,\ y > 0$ och $k \in \mathbb{R}$ gäller följande räknelagar:} (p.133)

\begin{align}
    {}^a\log xy &= {}^a\log x + {}^a\log y,\\
    {}^a\log \frac{x}{y} &= {}^a\log x - {}^a\log y,\\
    {}^a\log x^k &= k{}^a\log x.
\end{align}

\subsection{Trigonometriska funktioner (p.139)}

\textbf{Grader} (p.139)

\begin{align}
    \cos 45\degree &= \sin 45\degree = \frac{1}{\sqrt{2}}, &\tan 45\degree &= \cot 45\degree = 1,\\
    \cos 60\degree &= \sin 30\degree = \frac{1}{2},        &\tan 60\degree &= \cot 30\degree = \sqrt{3},\\
    \cos 30\degree &= \sin 60\degree = \frac{\sqrt{3}}{2}, &\tan 30\degree &= \cot 60\degree = \frac{1}{\sqrt{3}}.
\end{align}

\textbf{Radianer} (p.140)

\begin{align}
    \cos \frac{\pi}{4} &= \sin \frac{\pi}{4} = \frac{1}{\sqrt{2}}, &\tan \frac{\pi}{4} &= \cot \frac{\pi}{4} = 1,\\
    \cos \frac{\pi}{3} &= \sin \frac{\pi}{6} = \frac{1}{2},        &\tan \frac{\pi}{3} &= \cot \frac{\pi}{6} = \sqrt{3},\\
    \cos \frac{\pi}{6} &= \sin \frac{\pi}{3} = \frac{\sqrt{3}}{2}, &\tan \frac{\pi}{6} &= \cot \frac{\pi}{3} = \frac{1}{\sqrt{3}}.
\end{align}

\textbf{Tangens och cotangens} (p.142)

\begin{align}
    \tan x &= \frac{\sin x}{\cos x}, \quad x \neq \frac{\pi}{2} + k\pi,\\
    \nonumber \\
    \cot x &= \frac{\cos x}{\sin x}, \quad x \neq k\pi.
\end{align}

\subsection{Derivata av elementära funktioner (p.219)}

\begin{align}
    D\ e^x &= e^x,\\
    D\ \ln x &= \frac{1}{x},\\
    D\ a^x &= a^x ln a, \qquad a > 0\ \mathrm{konstant},\\
    D\ {}^{a}\log x &= \frac{1}{x \ln a}, \qquad a > 0,\ a \neq 1\ \mathrm{konstant},\\
    D\ x^\alpha &= \alpha x^{\alpha - 1}, \qquad \alpha\ \mathrm{konstant},\\
    D\ \sin x &= \cos x,\\
    D\ \cos x &= -\sin x,\\
    D\ \tan x &= \frac{1}{\cos^2 x},\\
    D\ \cot x &= -\frac{1}{\sin^2 x},\\
    D\ \arcsin x &= \frac{1}{\sqrt{1 - x^x}},\\
    D\ \arccos x &= -\frac{1}{\sqrt{1 - x^2}},\\
    D\ \arctan x &= \frac{1}{1 + x^2},\\
    D\ \mathrm{arccot} x &= -\frac{1}{1 + x^2}.
\end{align}

\subsection{Elementära primitiva funktioner (p.281)}

\begin{align}
    \int e^{x}\ dx &= e^x + C,\\
    \int \frac{1}{x}\ dx &= ln x + C,\\
    \int x^{\alpha}\ dx &= \frac{x^{\alpha + 1}}{\alpha + 1} + C \qquad (\alpha \neq -1),\\
    \int \cos x\ dx &= \sin x + C,\\
    \int \sin x\ dx &= -\cos x + C,\\
    \int \frac{1}{\cos^2 x}\ dx &= \tan x + C,\\
    \int \frac{1}{\sin^2 x}\ dx &= -\cot x + C,\\
    \int \frac{1}{\sqrt{1 - x^2}}\ dx &= \arcsin x + C,\\
    \int \frac{1}{1 - x^2}\ dx &= \arctan x + C.
\end{align}

\newpage
\section{Maclaurin- och Taylorutvecklingar}

\subsection{Maclaurinvecklingar}

\textbf{Definition 11.1} (Maclaurinpolynom). \textit{Låt $f$ vara en funktion som är (minst) $n$ gånger deriverbar i en omgivning av punkten 0. Polynomet}

\begin{equation}
    p_n(x) = f(0) + f'(0)x + \frac{f''(0)}{2}x^2 + \frac{f^{(3)}(0)}{3!}x^3 +\ ...\ + \frac{f^{(n)}(0)}{n!}x^n
    \label{eq:maclaurinpolynom}
\end{equation}

\textit{kallas \textbf{Maclaurinpolynomet} av \textbf{ordning} $n$ till $f$}. (p.258)

\subsubsection{Restermen}

\textbf{Sats 11.1} (Maclaurins formel). \textit{Antag att funktionen $f$ har kontinuerliga derivator (minst) till och med ordning $n + 1$ i en omgivning av punkten} 0. \textit{Då gäller det, för alla $x$ i denna omgivning, att}

\begin{equation}
    f(x) = f(0) + f'(0)x + \frac{f''(0)}{2!}x^2 + \frac{f^{(3)}(0)}{3!}x^3 +\ ...\ + \frac{f^{(n)}(0)}{n!}x^n + R_{n + 1}(xA,)
    \label{eq:maclaurinpolynom_och_rest}
\end{equation}

\textit{där}


\begin{equation}
    R_{n + 1}(x) = \frac{f^{(n + 1)(\epsilon)}}{(n +  1)!}x^{n + 1}
    \label{eq:maclaurinpolynom_rest}
\end{equation}

\textit{för något $\epsilon$ mellan $0$ och $x$}. (p.259)

\subsection{Taylorutvecklingar}

\textbf{Sats 11.4} (Talyors formel.) \textit{Antag att funktionen $f$ har kontinuerliga derivator (minst) till och med ordning $n + 1$ i en omgivning av punkten $a$. Då gäller det, för alla ¤x¤ i denna omgivning, att}

\begin{equation}
    f(x) = f(a) + f'(a)(x - a) + \frac{f''(a)}{2}(x - a)^2 +\ ...\ + \frac{f^{(n)}(a)}{n!}(x - a)^n + R_{n + 1}(x),
\end{equation}

\begin{equation}
    \text{där}\qquad R_{n + 1}(x) = \frac{f^{(n + 1)}(\beta)}{(n + 1)!}(x - a)^{n + 1}
\end{equation}

\textit{för något $\beta$ mellan $a$ och $x$.} (p.275)

\section{Integraler}

\subsection{Räknelagar för integraler (p.311)}

\textbf{Sats 13.4}. \textit{Antag att funktionerna $f$ och $g$ är integrerbara p[ intervallet $[a, b]$. Då gäller:}

\begin{align}
    \int_a^b{\alpha f(x)} dx &= \alpha \int_a^b{f(x)} dx, \qquad \alpha\ \text{konstant},\\
    \int_a^b{(f(x) + g(x))} dx &= \int_a^b{f(x)} dx + \int_a^b{g(x)} dx,\\
    \int_a^b{(f(x)} dx &= \int_a^c{f(x)} dx + \int_c^b{f(x)} dx,\\
    f(x) \leq g(x) \quad \text{då}\ a \leq x \leq b \quad &\Rightarrow \quad \int_a^b{f(x)} dx \leq \int_a^b{g(x)} dx.
\end{align}

\subsubsection{Partialintegration (p.319)}

\textbf{Sats 13.9} (Partialintegration). \textit{Antag att $F$ är en primitiv funktion till $f$, och att $f$ och $g'$ är kontinuerliga på intervallet $[a, b]$. Då gäller}

\begin{equation}
    \int_a^b{f(x)g(x)}dx = \left [F(x)g(x)\right ]_a^b - \int_a^b{F(x)g'(x)}dx.
\end{equation}

\end{document}

