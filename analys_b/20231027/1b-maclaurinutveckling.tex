\documentclass[11pt]{article}
\usepackage[utf8]{inputenc}
\usepackage[T1]{fontenc}     % Fix weird character
\usepackage{geometry}
\usepackage{amsmath}
\usepackage{amssymb}
\usepackage{gensymb}
\usepackage{spalign}
\usepackage{xfrac}
\usepackage{parskip}
\usepackage{float}  % figure[H]
\usepackage[style=ieee,backend=biber]{biblatex}
\usepackage[breaklinks=true,bookmarks=true,hidelinks]{hyperref}
\usepackage{tikz}

\geometry{
    a4paper,
    hmargin=2.54cm,
    tmargin=1.27cm,
    bmargin=1.27cm,
    includeheadfoot
}
\setcounter{secnumdepth}{0}  % Disable section numbering

\newcommand*\circled[1]{\tikz[baseline=(char.base)]{
            \node[shape=circle,draw,inner sep=1pt] (char) {#1};}}

\begin{document}
Med hjälp av Maclaurinutveckling beräkna följande gränsvärdet

\[
    \lim_{x \to 0} \frac{e^{2x} - 1 - \sin(2x)}{\cos(3x) - 1}.
\]

\noindent\rule{\textwidth}{0.5pt}

\textbf{Maclaurinutveckling restermen}

\[
    f(x) = f(0) + f'(0)x + \frac{f''(0)}{2!}x^2 + \frac{f^{(3)}(0)}{3!}x^3 +\ ...\ + \frac{f^{(n)}(0)}{n!}x^n + R_{n + 1}(xA,)
\]

\textit{där}


\[
    R_{n + 1}(x) = \frac{f^{(n + 1)}(\xi)}{(n +  1)!}x^{n + 1}
\]

\textit{för något $\xi$ mellan $0$ och $x$}. (p.259)

\textbf{Svagare form av resttermen}

\[
    B(x) = \frac{f^{(n + 1)}(\xi)}{(n + 1)!}
\]

Lösning:

\circled{1}.%
\begin{minipage}[t]{5cm}
    \setlength{\abovedisplayskip}{0pt}
    \setlength{\belowdisplayskip}{0pt}
    \setlength{\abovedisplayshortskip}{0pt}
    \setlength{\belowdisplayshortskip}{0pt}
    \begin{align*}
        f(x)  &= e^{2x} \Rightarrow f(0) &= 1\\
        f'(x) &= 2e^{2x} \Rightarrow f'(0) &= 2\\
        f''(x) &= 2e^{2x} \Rightarrow f''(0) &= 2\\
    \end{align*}
\end{minipage}

\circled{2}.%
\begin{minipage}[t]{5cm}
    \setlength{\abovedisplayskip}{0pt}
    \setlength{\belowdisplayskip}{0pt}
    \setlength{\abovedisplayshortskip}{0pt}
    \setlength{\belowdisplayshortskip}{0pt}
    \begin{align*}
        f(x)  &= \sin(2x) &\Rightarrow\quad f(0) = 0\\
        f'(x) &= 2\cos(2x) &\Rightarrow\quad f'(0) = 2\\
        f'''(x) &= -4\sin(2x) &\Rightarrow\quad  &f''(0) = 0\\
        f^{(3)}(x) &= -8\cos(2x) &\Rightarrow\quad f^{(3)}(0) = -8\\
        f^{(4)}(x) &= 16\sin(2x) &\Rightarrow\quad f^{(4)}(0) = 0\\
        f^{(5)}(x) &= 32\cos(2x) &\Rightarrow\quad f^{(5)}(0) = 32
    \end{align*}
\end{minipage}

\end{document}

